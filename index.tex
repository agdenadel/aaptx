%**************************************%
%* Generated from MathBook XML source *%
%*    on 2017-08-07T16:07:41-07:00    *%
%*                                    *%
%*   http://mathbook.pugetsound.edu   *%
%*                                    *%
%**************************************%
\documentclass[10pt,]{book}
%% Custom Preamble Entries, early (use latex.preamble.early)
%% Inline math delimiters, \(, \), need to be robust
%% 2016-01-31:  latexrelease.sty  supersedes  fixltx2e.sty
%% If  latexrelease.sty  exists, bugfix is in kernel
%% If not, bugfix is in  fixltx2e.sty
%% See:  https://tug.org/TUGboat/tb36-3/tb114ltnews22.pdf
%% and read "Fewer fragile commands" in distribution's  latexchanges.pdf
\IfFileExists{latexrelease.sty}{}{\usepackage{fixltx2e}}
%% Text height identically 9 inches, text width varies on point size
%% See Bringhurst 2.1.1 on measure for recommendations
%% 75 characters per line (count spaces, punctuation) is target
%% which is the upper limit of Bringhurst's recommendations
%% Load geometry package to allow page margin adjustments
\usepackage{geometry}
\geometry{letterpaper,total={340pt,9.0in}}
%% Custom Page Layout Adjustments (use latex.geometry)
%% This LaTeX file may be compiled with pdflatex, xelatex, or lualatex
%% The following provides engine-specific capabilities
%% Generally, xelatex and lualatex will do better languages other than US English
%% You can pick from the conditional if you will only ever use one engine
\usepackage{ifthen}
\usepackage{ifxetex,ifluatex}
\ifthenelse{\boolean{xetex} \or \boolean{luatex}}{%
%% begin: xelatex and lualatex-specific configuration
%% fontspec package will make Latin Modern (lmodern) the default font
\ifxetex\usepackage{xltxtra}\fi
\usepackage{fontspec}
%% realscripts is the only part of xltxtra relevant to lualatex 
\ifluatex\usepackage{realscripts}\fi
%% 
%% Extensive support for other languages
\usepackage{polyglossia}
\setdefaultlanguage{english}
%% Magyar (Hungarian)
\setotherlanguage{magyar}
%% Spanish
\setotherlanguage{spanish}
%% Vietnamese
\setotherlanguage{vietnamese}
%% end: xelatex and lualatex-specific configuration
}{%
%% begin: pdflatex-specific configuration
%% translate common Unicode to their LaTeX equivalents
%% Also, fontenc with T1 makes CM-Super the default font
%% (\input{ix-utf8enc.dfu} from the "inputenx" package is possible addition (broken?)
\usepackage[T1]{fontenc}
\usepackage[utf8]{inputenc}
%% end: pdflatex-specific configuration
}
%% Monospace font: Inconsolata (zi4)
%% Sponsored by TUG: http://levien.com/type/myfonts/inconsolata.html
%% See package documentation for excellent instructions
%% One caveat, seem to need full file name to locate OTF files
%% Loads the "upquote" package as needed, so we don't have to
%% Upright quotes might come from the  textcomp  package, which we also use
%% We employ the shapely \ell to match Google Font version
%% pdflatex: "varqu" option produces best upright quotes
%% xelatex,lualatex: add StylisticSet 1 for shapely \ell
%% xelatex,lualatex: add StylisticSet 2 for plain zero
%% xelatex,lualatex: we add StylisticSet 3 for upright quotes
%% 
\ifthenelse{\boolean{xetex} \or \boolean{luatex}}{%
%% begin: xelatex and lualatex-specific monospace font
\usepackage{zi4}
\setmonofont[BoldFont=Inconsolatazi4-Bold.otf,StylisticSet={1,3}]{Inconsolatazi4-Regular.otf}
%% end: xelatex and lualatex-specific monospace font
}{%
%% begin: pdflatex-specific monospace font
\usepackage[varqu]{zi4}
%% end: pdflatex-specific monospace font
}
%% Symbols, align environment, bracket-matrix
\usepackage{amsmath}
\usepackage{amssymb}
%% allow more columns to a matrix
%% can make this even bigger by overriding with  latex.preamble.late  processing option
\setcounter{MaxMatrixCols}{30}
%%
%% Color support, xcolor package
%% Always loaded.  Used for:
%% mdframed boxes, add/delete text, author tools
\PassOptionsToPackage{usenames,dvipsnames,svgnames,table}{xcolor}
\usepackage{xcolor}
%%
%% Semantic Macros
%% To preserve meaning in a LaTeX file
%% Only defined here if required in this document
%% Used for warnings, typically bold and italic
\newcommand{\alert}[1]{\textbf{\textit{#1}}}
%% Used for inline definitions of terms
\newcommand{\terminology}[1]{\textbf{#1}}
%% Subdivision Numbering, Chapters, Sections, Subsections, etc
%% Subdivision numbers may be turned off at some level ("depth")
%% A section *always* has depth 1, contrary to us counting from the document root
%% The latex default is 3.  If a larger number is present here, then
%% removing this command may make some cross-references ambiguous
%% The precursor variable $numbering-maxlevel is checked for consistency in the common XSL file
\setcounter{secnumdepth}{3}
%% Environments with amsthm package
%% Theorem-like environments in "plain" style, with or without proof
\usepackage{amsthm}
\theoremstyle{plain}
%% Numbering for Theorems, Conjectures, Examples, Figures, etc
%% Controlled by  numbering.theorems.level  processing parameter
%% Always need a theorem environment to set base numbering scheme
%% even if document has no theorems (but has other environments)
\newtheorem{theorem}{Theorem}[section]
%% Only variants actually used in document appear here
%% Style is like a theorem, and for statements without proofs
%% Numbering: all theorem-like numbered consecutively
%% i.e. Corollary 4.3 follows Theorem 4.2
\newtheorem{corollary}[theorem]{Corollary}
%% Definition-like environments, normal text
%% Numbering is in sync with theorems, etc
\theoremstyle{definition}
\newtheorem{definition}[theorem]{Definition}
%% Remark-like environments, normal text
%% Numbering is in sync with theorems, etc
\theoremstyle{definition}
\newtheorem{remark}[theorem]{Remark}
\newtheorem{note}[theorem]{Note}
\newtheorem{warning}[theorem]{Warning}
%% Example-like environments, normal text
%% Numbering is in sync with theorems, etc
\theoremstyle{definition}
\newtheorem{example}[theorem]{Example}
%% Miscellaneous environments, normal text
%% Numbering for inline exercises and lists is in sync with theorems, etc
\theoremstyle{definition}
\newtheorem{exercise}[theorem]{Exercise}
%% Localize LaTeX supplied names (possibly none)
\renewcommand*{\proofname}{Proof}
\renewcommand*{\appendixname}{Appendix}
\renewcommand*{\chaptername}{Chapter}
%% Equation Numbering
%% Controlled by  numbering.equations.level  processing parameter
\numberwithin{equation}{section}
%% For improved tables
\usepackage{array}
%% Some extra height on each row is desirable, especially with horizontal rules
%% Increment determined experimentally
\setlength{\extrarowheight}{0.2ex}
%% Define variable thickness horizontal rules, full and partial
%% Thicknesses are 0.03, 0.05, 0.08 in the  booktabs  package
\makeatletter
\newcommand{\hrulethin}  {\noalign{\hrule height 0.04em}}
\newcommand{\hrulemedium}{\noalign{\hrule height 0.07em}}
\newcommand{\hrulethick} {\noalign{\hrule height 0.11em}}
%% We preserve a copy of the \setlength package before other
%% packages (extpfeil) get a chance to load packages that redefine it
\let\oldsetlength\setlength
\newlength{\Oldarrayrulewidth}
\newcommand{\crulethin}[1]%
{\noalign{\global\oldsetlength{\Oldarrayrulewidth}{\arrayrulewidth}}%
\noalign{\global\oldsetlength{\arrayrulewidth}{0.04em}}\cline{#1}%
\noalign{\global\oldsetlength{\arrayrulewidth}{\Oldarrayrulewidth}}}%
\newcommand{\crulemedium}[1]%
{\noalign{\global\oldsetlength{\Oldarrayrulewidth}{\arrayrulewidth}}%
\noalign{\global\oldsetlength{\arrayrulewidth}{0.07em}}\cline{#1}%
\noalign{\global\oldsetlength{\arrayrulewidth}{\Oldarrayrulewidth}}}
\newcommand{\crulethick}[1]%
{\noalign{\global\oldsetlength{\Oldarrayrulewidth}{\arrayrulewidth}}%
\noalign{\global\oldsetlength{\arrayrulewidth}{0.11em}}\cline{#1}%
\noalign{\global\oldsetlength{\arrayrulewidth}{\Oldarrayrulewidth}}}
%% Single letter column specifiers defined via array package
\newcolumntype{A}{!{\vrule width 0.04em}}
\newcolumntype{B}{!{\vrule width 0.07em}}
\newcolumntype{C}{!{\vrule width 0.11em}}
\makeatother
%% Figures, Tables, Listings, Floats
%% The [H]ere option of the float package fixes floats in-place,
%% in deference to web usage, where floats are totally irrelevant
%% We re/define the figure, table and listing environments, if used
%%   1) New mbxfigure and/or mbxtable environments are defined with float package
%%   2) Standard LaTeX environments redefined to use new environments
%%   3) Standard LaTeX environments redefined to step theorem counter
%%   4) Counter for new environments is set to the theorem counter before caption
%% You can remove all this figure/table setup, to restore standard LaTeX behavior
%% HOWEVER, numbering of figures/tables AND theorems/examples/remarks, etc
%% WILL ALL de-synchronize with the numbering in the HTML version
%% You can remove the [H] argument of the \newfloat command, to allow flotation and 
%% preserve numbering, BUT the numbering may then appear "out-of-order"
\usepackage{float}
\usepackage[bf]{caption} % http://tex.stackexchange.com/questions/95631/defining-a-new-type-of-floating-environment 
\usepackage{newfloat}
% Figure environment setup so that it no longer floats
\SetupFloatingEnvironment{figure}{fileext=lof,placement={H},within=section,name=Figure}
% figures have the same number as theorems: http://tex.stackexchange.com/questions/16195/how-to-make-equations-figures-and-theorems-use-the-same-numbering-scheme 
\makeatletter
\let\c@figure\c@theorem
\makeatother
% Table environment setup so that it no longer floats
\SetupFloatingEnvironment{table}{fileext=lot,placement={H},within=section,name=Table}
% tables have the same number as theorems: http://tex.stackexchange.com/questions/16195/how-to-make-equations-figures-and-theorems-use-the-same-numbering-scheme 
\makeatletter
\let\c@table\c@theorem
\makeatother
%% Raster graphics inclusion, wrapped figures in paragraphs
%% \resizebox sometimes used for images in side-by-side layout
\usepackage{graphicx}
%%
%% Multiple column, column-major lists
\usepackage{multicol}
%% More flexible list management, esp. for references and exercises
%% But also for specifying labels (i.e. custom order) on nested lists
\usepackage{enumitem}
%% Lists of references in their own section, maximum depth 1
\newlist{referencelist}{description}{4}
\setlist[referencelist]{leftmargin=!,labelwidth=!,labelsep=0ex,itemsep=1.0ex,topsep=1.0ex,partopsep=0pt,parsep=0pt}
%% Lists of exercises in their own section, maximum depth 4
\newlist{exerciselist}{description}{4}
\setlist[exerciselist]{leftmargin=0pt,itemsep=1.0ex,topsep=1.0ex,partopsep=0pt,parsep=0pt}
%% Package for tables spanning several pages
\usepackage{longtable}
%% hyperref driver does not need to be specified
\usepackage{hyperref}
%% configure hyperref's  \url  to match listings' inline verbatim
\renewcommand\UrlFont{\small\ttfamily}
%% Hyperlinking active in PDFs, all links solid and blue
\hypersetup{colorlinks=true,linkcolor=blue,citecolor=blue,filecolor=blue,urlcolor=blue}
\hypersetup{pdftitle={First-Semester Abstract Algebra}}
%% If you manually remove hyperref, leave in this next command
\providecommand\phantomsection{}
%% Graphics Preamble Entries
\usepackage{tikz}
\usepackage{tkz-graph}
\usepackage{tkz-euclide}
\usetikzlibrary{patterns}
\usetikzlibrary{positioning}
\usetikzlibrary{matrix,arrows}
\usetikzlibrary{calc}
\usetikzlibrary{shapes}
\usetikzlibrary{through,intersections,decorations,shadows,fadings}

\usepackage[all]{xy}
\usepackage{pgfplots}
%% If tikz has been loaded, replace ampersand with \amp macro
%% NB: calc redefines \setlength
\usepackage{calc}
%% used repeatedly for vertical dimensions of sidebyside panels
\newlength{\panelmax}
%% extpfeil package for certain extensible arrows,
%% as also provided by MathJax extension of the same name
%% NB: this package loads mtools, which loads calc, which redefines
%%     \setlength, so it can be removed if it seems to be in the 
%%     way and your math does not use:
%%     
%%     \xtwoheadrightarrow, \xtwoheadleftarrow, \xmapsto, \xlongequal, \xtofrom
%%     
%%     we have had to be extra careful with variable thickness
%%     lines in tables, and so also load this package late
\usepackage{extpfeil}
%% Custom Preamble Entries, late (use latex.preamble.late)
%% Begin: Author-provided packages
%% (From  docinfo/latex-preamble/package  elements)
%% End: Author-provided packages
%% Begin: Author-provided macros
%% (From  docinfo/macros  element)
%% Plus three from MBX for XML characters
\def\Z{\mathbb{Z}}
\def\zn{\mathbb{Z}_n}
\def\znc{\mathbb{Z}_n^\times}
\def\R{\mathbb{R}}
\def\Q{\mathbb{Q}}
\def\C{\mathbb{C}}
\def\N{\mathbb{N}}
\def\M{\mathbb{M}}
\def\G{\mathcal{G}}
\def\0{\mathbf 0}
\def\Gdot{\langle G, \cdot\,\rangle}
\def\phibar{\overline{\phi}}
\DeclareMathOperator{\lcm}{lcm}
\DeclareMathOperator{\Ker}{Ker}
\def\siml{\sim_L}
\def\simr{\sim_R}
\newcommand{\lt}{ < }
\newcommand{\gt}{ > }
\newcommand{\amp}{ & }
%% End: Author-provided macros
%% Title page information for book
\title{First-Semester Abstract Algebra\\
{\large A Structural Approach}}
\author{Jessica K. Sklar\\
Department of Mathematics\\
Pacific Lutheran University\\
\href{mailto:sklarjk@plu.edu}{\nolinkurl{sklarjk@plu.edu}}
}
\date{}
\begin{document}
\frontmatter
%% begin: half-title
\thispagestyle{empty}
{\centering
\vspace*{0.28\textheight}
{\Huge First-Semester Abstract Algebra}\\[2\baselineskip]
{\LARGE A Structural Approach}\\
}
\clearpage
%% end:   half-title
%% begin: adcard
\thispagestyle{empty}
\null%
\clearpage
%% end:   adcard
%% begin: title page
%% Inspired by Peter Wilson's "titleDB" in "titlepages" CTAN package
\thispagestyle{empty}
{\centering
\vspace*{0.14\textheight}
%% Target for xref to top-level element is ToC
\addtocontents{toc}{\protect\hypertarget{index}{}}
{\Huge First-Semester Abstract Algebra}\\[\baselineskip]
{\LARGE A Structural Approach}\\[3\baselineskip]
{\Large Jessica K. Sklar}\\[0.5\baselineskip]
{\Large Pacific Lutheran University}\\}
\clearpage
%% end:   title page
%% begin: copyright-page
\thispagestyle{empty}
\vspace*{\stretch{2}}
\noindent{\bf Edition}: First Edition\par\medskip
\noindent\textcopyright\ 2017\quad{}Jessica K. Sklar\\[0.5\baselineskip]
Permission is granted to copy, distribute and/or modify this document under the terms of the GNU Free Documentation License, Version 1.3 or any later version published by the Free Software Foundation; with no Invariant Sections, no Front-Cover Texts, and no Back-Cover Texts.  A copy of the license is included in the appendix entitled ``GNU Free Documentation License.''\par\medskip
\vspace*{\stretch{1}}
\null\clearpage
%% end:   copyright-page
%% begin: acknowledgement
\chapter*{Acknowledgements}\label{acknowledgement-1}
\addcontentsline{toc}{chapter}{Acknowledgements}
Thank you to Jennifer Nordstrom of Linfield College for introducing me to the \href{http://mathbook.pugetsound.edu}{PreTeXt} authoring system; to Rob Beezer at the University of Puget Sound for facilitating my entry into the PreTeXt world;  to David Farmer at the \href{https://aimath.org/textbooks/}{ Open Textbook Initiative} for typesetting the initial draft of this book in PreTeXt; and to Rob, David, Mitch Keller, Alex Jordan, Bob Plantz, and Alex Best and everyone else otheir extensive technical help via the \href{https://groups.google.com/forum/?fromgroups\#!forum/pretext-support}{PreTeXt support} group.%
%% end:   acknowledgement
%% begin: preface
\chapter*{Preface}\label{preface-1}
\addcontentsline{toc}{chapter}{Preface}

    At its most basic level, abstract algebra is the study of structures. Just as an architect may examine buildings or an anthropologist societal hierarchies, an algebraist explores the nature of sets equipped with binary operations that satisfy certain properties. While these structures may not seem at first to be very important, they are at the heart of most, if not all, mathematical endeavors. On an elemental level, they allow us to solve systems of equations; on a more global-level, they are behind some of our most important cryptographic systems. We even use them implicitly when telling time!
  %
\par

    Our focus in this course will be exploring some of the most fundamental algebraic structures: namely, groups, rings, and fields. Along the way, we will explore rigorous mathematical notions of similarity and difference: When can we consider two objects to be more or less ``the same''? When are they fundamentally different? For instance, consider two houses that have exactly the same construction, but are painted different colors. Are they the same house? No. But viewed structurally (as opposed to aesthetically) they are the same. This means that if we know certain information about one of the houses (say, how far the bathroom is from the kitchen) we know the same information about the other house. However, knowing that the first house is painted yellow does not tell us anything about the second house's color. We explore an analogous idea in mathematics, namely, the concept of \emph{isomorphism}.
  %
\par

    Along the way, we will gain experience writing mathematical proofs and will see plenty of specific examples demonstrating more general ideas.
  %
%% end:   preface
%% begin: table of contents
%% Adjust Table of Contents
\setcounter{tocdepth}{1}
\renewcommand*\contentsname{Contents}
\tableofcontents
%% end:   table of contents
\mainmatter
\typeout{************************************************}
\typeout{Chapter 1 Preliminaries}
\typeout{************************************************}
\chapter[{Preliminaries}]{Preliminaries}\label{pre}
\typeout{************************************************}
\typeout{Section 1.1 Sets}
\typeout{************************************************}
\section[{Sets}]{Sets}\label{section-1}

    We start off by providing a ``definition'' of a basic mathematical structure to which we will soon add bells and whistles. We use quotes here because what follows doesn't have the precision we usually require when defining a mathematical object.
  %
\begin{definition}[{}]\label{definition-1}
 A \terminology{set} is an (unordered)
    collection of objects.
  %
\end{definition}
\par

    This is only sort of  a ``definition'' because it is not a
    rigorous definition of a set. For instance, what do we mean by a
    ``collection'' of objects? This ``definition'' will be sufficient for
    our course, but be warned that defining a set in this vague way can
    lead to some serious mathematical issues, such as \emph{Russell's
    paradox}.  A mathematician whose expertise is in set
    theory may scowl disagreeably if you try to define a set as we have
    above.
  %
\begin{note}[]\label{note-1}

    Let \(S\) be the set of all sets that aren't members
    of themselves.  Is \(S\) a member of itself? If you think carefully
    about this, you'll see that \(S\) can be neither a member of itself,
    nor \emph{not} a member of itself. Uh oh!  This contradiction
    is known as ``Russell's paradox'' (named for the British philosopher, mathematician, and all-round academic Bertrand Russell). Mathematicians deal with this by
    declaring that some object collections, called \emph{classes}, are
    not in fact sets.%
\end{note}
\begin{definition}[{}]\label{definition-2}

        The members of a set are called its \terminology{elements}. If
        \(S\) is a set, we write \(x\in S\) to indicate ``\(x\) is an element
        of \(S\),'' and \(x \not\in S\) to indicate ``\(x\) is not an element
        of \(S\).'' There is a unique set containing no elements; it is
        called the \terminology{empty set}, and denoted by \(\emptyset\).
      %
\label{notation-1}
\label{notation-2}
\label{notation-3}
\end{definition}
\par

    Sets must be \emph{well defined}: that is, it must be clear
    exactly which objects are in a set, and which objects aren't.
    For instance, the set of all integers is well defined, but the
    set of all big integers is not well defined, since it is
    unclear what ``big'' means in this context.
  %
\par

    We refer to some sets so frequently in mathematics that we have special notation for them.
  %
\begin{example}[]\label{example-1}

\leavevmode%
\begin{description}
\item[{\(\Z\)}]\hypertarget{li-1}{}the set of all integers  (the \(\Z\) comes ``zahlen,'' the German word for ``numbers'')%
\item[{\(\Q\)}]\hypertarget{li-2}{}set of all rational numbers%
\item[{\(\R\)}]\hypertarget{li-3}{}the set of all real numbers%
\item[{\(\C\)}]\hypertarget{li-4}{}the set of all complex numbers%
\item[{\(\N\)}]\hypertarget{li-5}{}the set of all natural numbers, i.e., \(\{0, 1, 2, \ldots\}\)%
\par
 (Be aware that many books/mathematicians do not include \(0\) in the set of natural numbers%
\end{description}

We can further write denote by \(\Z^+\) the set of all positive integers, and by \(\Z^*\) the set of all nonzero integers.
Can you guess what the penultimate notation represents if we replace \(\Z\) with \(\Q\) or \(\R\), and/or \(+\) with \(-\)? What about 
what the last notation, if we replace \(\Z\) with \(\Q\), \(\R\), or \(\C\)?
%
\label{notation-4}
\label{notation-5}
\label{notation-6}
\label{notation-7}
\label{notation-8}
\label{notation-9}
\label{notation-10}
\label{notation-11}
\end{example}
\par

    We also provide notation for commonly considered sets of matrices:
  %
\begin{definition}[{}]\label{definition-3}

 Given \(m,n\in \Z^+\) and a set \(S\), we define \(\M_{m\times n}(S)\) to
    be the set of all \(m\times n\) matrices over \(S\) (that is, of all \(m\times n\) matrices with entries in \(S\)). We use the
    shorthand notation \(\M_n(S)\) for the set \(\M_{n\times n}(S)\).\label{notation-12}
\label{notation-13}
\end{definition}
\par

    One common way of describing a set is to list its elements in
    curly braces, separated by commas; you can use ellipses to
    indicate a repeated pattern of elements. A few examples are
    \(\{1,4,\pi\}\), \(\{3, 4, 5, \ldots\}\), and \(\{\ldots, -4, -2, 0,
    2, 4, \ldots\}\); the last of these can be written more concisely as \(\{0,\pm 2, \pm 4,\ldots\}\). Note that since elements of a set are
    unordered, the sets \(\{1,4,\pi\}\) and \(\{4,\pi, 1\}\), for
    instance, are identical.
  %
\par

    Another method is using \terminology{set-builder notation}. This consists of an element name (or
    names), followed by a colon (meaning ``such that''), followed by
    a Boolean expression involving the element name(s), all
    surrounded by curly braces.%
\begin{warning}[]\label{warning-1}
The use of a colon to denote ``such that'' is \emph{only} valid in the above set-builder notation context.  Outside of this context, you should never use a colon to denote ``such that''; instead, use the abbreviation ``s.t.'' or write out the actual words.   Conversely, never use one of those ways of indicating ``such that'' within set-builder notation; always use a colon there.  Why?  Convention.%
\end{warning}
\par

    For example, %
\begin{equation*}
 \{x\in \Z : x > 4\} 
\end{equation*}
 is the set \(\{5, 6, 7, \ldots\}\), while %
\begin{equation*}
 \{z\in \C : |z|=1\}
\end{equation*}
 is the set of all complex numbers at distance 1 from the origin in the complex plane.
  %
\begin{note}[]\label{note-2}

    If one simply writes \(\{x\,:\,x>4\}\),
    it is unclear what this set is; it could be the set of all
    integers greater than 4, or the set of all real numbers greater
    than 4, or something else. When one can, it is better to
    identify the named element(s) as a member (members) of a known
    set, such as \(\R\) or \(\Z\), whenever possible.
  %
\end{note}
\begin{definition}[{}]\label{definition-4}

        Set \(A\) is a \terminology{subset} of \(B\) (and set \(B\) is a \terminology{superset} of \(A\)) if every element in \(A\) is also in \(B\). We
        denote ``\(A\) is a subset of \(B\)'' by \(A\subseteq B\). Sets \(A\)
        and \(B\) are said to be \terminology{equal}, and we write \(A=B\), if they
        contain exactly the same elements; equivalently, \(A=B\) if and
        only if \(A \subseteq B\) and \(B\subseteq A\). Set \(A\) is a \terminology{proper} subset of set \(B\) if \(A\subseteq B\) but \(A\neq B\); we
        write this \(A\subsetneq B\).
        %
\label{notation-14}
\label{notation-15}
\end{definition}
\begin{remark}[]\label{remark-1}
 Sometimes the notation \(A\subset B\) is  used to
        indicate that \(A\) is a proper subset of \(B\), and sometimes it is simply used to mean that \(A\) is a subset\textemdash{}proper or improper\textemdash{}of \(B\). We will not use the notation \(A \subset B\) in this text. %
\end{remark}
\begin{note}[]\label{note-3}

      One often proves that two sets \(A\) and
      \(B\) are equal by proving that \(A\subseteq B\) and \(B\subseteq A\).
    %
\end{note}
\begin{example}[]\label{example-2}

        We have the following: \(\Z^+ \subseteq \Z \subseteq \Q \subseteq \R \subseteq \C\).
      %
\end{example}
\begin{definition}[{}]\label{definition-5}

        The \terminology{power set} of \(A\), denoted \(P(A)\), is the set of
        all subsets of \(A\). (Note that the power set of any set
        contains the empty set as an element.)
      %
\label{notation-16}
\end{definition}
\begin{warning}[]\label{warning-2}

      Be careful to use your curly braces correctly when writing power sets!
      Remember, the power set of a set is a \emph{set of sets}.
    %
\end{warning}
\par

    The following provides a good example of using braces correctly.
  %
\begin{example}[]\label{example-3}

        If \(A=\{a,b\}\), then \(P(A)=\{\emptyset, \{a\}, \{b\}, \{a,b\}\}\). Note that the element \(\{a,b\}\) of \(P(A)\) could also be written simply as \(A\).
      %
\end{example}
\begin{definition}[{}]\label{definition-6}
\leavevmode%
\begin{enumerate}
\item\hypertarget{li-6}{}
            If \(A\) and \(B\) are sets, then the \terminology{union} of \(A\) and \(B\), denoted \(A\cup B\), is the set
            \(A\cup B=\{x: x\in A \text{ or }  x\in B\};\) the \terminology{intersection} of \(A\) and \(B\), denoted \(A\cap B\), is the set
            \(A\cap B=\{x: x\in A \text{ and }  x\in B\};\) and the \terminology{difference} of \(A\) and \(B\), denoted \(A-B\) (or \(A\setminus B\)), is the set
            \(A-B=\{x: x\in A \text{ and }  x\not\in B\}.\)
          %
\item\hypertarget{li-7}{}
            More generally, given any collection of sets \(A_i\) (\(i\) in some index set \(I\)), the \terminology{union} of the \(A_i\) is %
\begin{equation*}
\bigcup_{i\in I}A_i=\{x: x\in A_i \text{ for some }  i\in I\},
\end{equation*}
 and the \terminology{intersection} of the \(A_i\) is %
\begin{equation*}
\bigcap_{i\in I}A_i=\{x: x\in A_i \text{ for every }  i\in I\}.
\end{equation*}

          %
\item\hypertarget{li-8}{}
            Sets \(A\) and \(B\) are \emph{disjoint} if \(A\cap B=\emptyset\).  More generally, sets \(A_i\) (\(i\) in some index set \(I\)) are
            \emph{disjoint} if %
\begin{equation*}
\bigcap_{i\in I}A_i=\emptyset
\end{equation*}
 and are \emph{mutually disjoint} if%
\begin{equation*}
A_i\cap A_j=\emptyset \text{ for all } i\neq j \in I. 
\end{equation*}

          %
\end{enumerate}
\label{notation-17}
\label{notation-18}
\label{notation-19}
\label{notation-20}
\label{notation-21}
\end{definition}
\par

    Notice that for any sets \(A\) and \(B\), \(A\cap B \subseteq A \subseteq A\cup B\) and
    \(A\cap B \subseteq B \subseteq A\cup B\). Also note that if sets \(A_i\) (\(i \in I\)) are mutually disjoint then they are also disjoint, but they may be disjoint without being mutually disjoint. For example, the sets \(\{i, i+1\}\) for \(i\in \Z\) are disjoint but not mutually disjoint. (Do you see why?)
  %
\par

    We define one more way of ``combining'' sets.
  %
\begin{definition}[{}]\label{definition-7}

        Let \(A\) and \(B\) be sets. Then the \terminology{direct product} (or \terminology{Cartesian 
        product}) \(A\times B\) of \(A\) and \(B\) is the set
        %
\begin{equation*}
A\times B =\{(a,b): \text{\(a\in A\), \(b\in B\)} \}.
\end{equation*}

      %
\par

        An element \((a,b)\) of
        \(A\times B\) is called an \terminology{ordered pair}. More generally, if
        \(A_1, A_2, \ldots, A_n\) are sets for some \(n\in \Z^+\), then the
        \terminology{product} of the \(A_i\) is
        %
\begin{equation*}
A_1\times A_2 \times \cdots \times A_n=\{(a_1, a_2 \ldots, a_n): a_i \in A_i \text{ for
          } i=1,2, \ldots, n\}; 
\end{equation*}

        the elements \((a_1,a_2,\ldots,a_n)\) of this
        product are called \(n\)-tuples (or triples, if \(n=3\)). (You can also have products of infinitely many sets, but we will
        not discuss that in this course.) Finally, if each set \(A_i\) is
        the same set \(A\), we can use the notation \(A^n\) to denote the
        product
        %
\begin{equation*}
A\times A \times \cdots \times A
\end{equation*}

        of \(n\) copies of
        \(A\).
      %
\label{notation-22}
\end{definition}
\begin{example}[]\label{example-4}

        For example, the Cartesian plane is the set \(\R^2\), and the set \(\Z \times \R\) consists of exactly the points in the plane with integer \(x\)-coordinates
        (that is, the points that lie on vertical lines intersecting the \(x\)-axis at integer values).
      %
\end{example}
\typeout{************************************************}
\typeout{Section 1.2 Functions}
\typeout{************************************************}
\section[{Functions}]{Functions}\label{section-2}

    You have probably encountered functions before. In introductory
    calculus, for instance, you typically deal with functions from
    \(\R\) to \(\R\) (e.g., the function \(f(x)=x^2\)). More generally,
    functions ``send'' elements of one set to elements of another set;
    these sets may or may not be sets of real numbers. We provide below
    a ``good enough for government work'' definition of a function.
  %
\begin{definition}[{}]\label{definition-8}

        A \terminology{function} \(f\) from a set \(S\) to a set \(T\) is a
        ``rule'' that assigns to each element \(s\) in \(S\) a unique
        element \(f(s)\) in \(T\); the element \(f(s)\) is called the \terminology{image of \(s\) under \(f\)}. If \(f\) is a function from \(S\) to \(T\),
        we write \(f: S \to T\), and call \(S\) the \terminology{domain} of \(f\) and
        \(T\) the \terminology{codomain} of \(f\). The \terminology{range} of \(f\) is
          %
\begin{equation*}
f(S)=\{f(s) \in T : s \in S\} \subseteq T.
\end{equation*}

      %
\par

        More generally, if \(U \subseteq S\), the \terminology{image of \(U\) in \(T\) under \(f\)} is
        %
\begin{equation*}
f(U)=\{f(u)\in T : u\in U\}.
\end{equation*}

      %
\par

        If \(V\subseteq T\), the \terminology{preimage of \(V\) in \(S\) under \(f\)} is the set
        %
\begin{equation*}
f^{\leftarrow}(V)=\{s\in S: f(s)\in V\}.
\end{equation*}

      %
\label{notation-23}
\label{notation-24}
\label{notation-25}
\end{definition}
\begin{example}[]\label{example-5}

        Consider the function \(f: \Z \to \R\) defined by \(f(x)=x^2\).
        The domain of \(f\) is \(\Z\) and the codomain of \(f\) is \(\R\); the
        range of \(f\) is \(\{x^2\,:\,x\in \Z\}=\{0,1,4,9,\ldots\}\). The image
        of \(\{-2,-1,1,2\}\) under \(f\) is the two-element set \(\{1,4\} \subseteq
        \R\), and the preimage of \(\{4,25,\pi\}\) under \(f\) is the set
        \(\{\pm 2, \pm 5\}\). (Do you see why \(\pm \sqrt{\pi}\) are not in this
        preimage?) What is the preimage of just \(\{\pi\}\) under \(f\)?
      %
\end{example}
\par

    The following definitions will be very important in our future work.
  %
\begin{definition}[{}]\label{definition-9}

        Let \(S\) and \(T\) be sets, and \(f:S\to T\).
        \leavevmode%
\begin{enumerate}
\item\hypertarget{li-9}{}
              Function \(f\) is \terminology{one-to-one} (1-1) if whenever \(s_1, s_2\in S\) with \(f(s_1)=f(s_2)\), we have \(s_1=s_2\).  Equivalently, \(f\) is one-to-one if
              whenever \(s_1\neq s_2 \in S\), then \(f(s_1)\neq f(s_2) \in T\).
            %
\item\hypertarget{li-10}{}
              Function \(f\) is \terminology{onto} if for every \(t\in T\), there exists an element \(s\in S\) such that \(f(s)=t\).  Equivalently, \(f\) is onto if \(f(S)=T\).
            %
\item\hypertarget{li-11}{}
              Function \(f\) is a \terminology{bijection} if it is both one-to-one and onto.
            %
\end{enumerate}

      %
\end{definition}
\begin{remark}[]\label{remark-2}

    We will often have to show functions are one-to-one or onto. Given a function \(f:S\to T\), the following methods are recommended.
%
\leavevmode%
\begin{itemize}[label=\textbullet]
\item{}
To prove that \(f\) is one-to-one: Let \(s_1,s_2 \in S\) with \(f(s_1)=f(s_2)\) and prove that then \(s_1=s_2\).%
\par

\alert{Note: It is \emph{not} sufficient to prove that if \(s_1=s_2\) in \(S\) then \(f(s_1)=f(s_2)\); that holds true for ANY function from \(S\) to \(T\)!  Be careful to \emph{assume} and \emph{prove} the correct facts.}%
\item{}
To prove that \(f\) is \emph{not} one-to-one: Identify two elements \(s_1 \neq s_2\) of \(S\) such that \(f(s_1)=f(s_2)\).
%
\item{}
To prove that \(f\) is onto: Let \(t\in T\) and prove that there exists an element \(s\in S\) with \(f(s)=t\).%
\par
\alert{Note: It is \emph{not} sufficient to prove that if \(s\in S\) then \(f(s)\) is in \(T\); that holds true for ANY function from \(S\) to \(T\)!  Again, be careful to \emph{assume} and \emph{prove} the correct facts.}%
\item{}
To prove that \(f\) is \emph{not} onto: Identify an element \(t\in T\) for which there is no \(s\in S\) with \(f(s)=t\).
%
\end{itemize}
\end{remark}
\begin{example}[]\label{example-6}

        Consider the function \(f: \R^* \to \R^+\) defined by \(f(x)=x^2\). Function \(f\) is \emph{not} one-to-one: indeed,
        \(-1\) and \(1\) are in \(\R^*\) with \(f(-1)=1=f(1)\) in \(\R^+\).
        However, \(f\) \emph{is} onto: indeed, let \(t\in \R^+\). Then
        \(\sqrt{t} \in \R^*\) with \(t=f(\sqrt{t})\), so we're done.
      %
\end{example}
\begin{example}[]\label{example-7}

        Consider the function \(f: \Z^+ \to \R\) defined by \(f(x)=x/2\). Function \(f\) \emph{is} one-to-one: indeed, let \(s_1, s_2 \in \Z^+\) with \(f(s_1)=f(s_2)\).
        Then \(s_1/2=f(s_1)=f(s_2)=s_2/2\); multiplying both sides of the equation \(s_1/2=s_2/2\) by 2, we obtain \(s_1=s_2\). However, \(f\) is \emph{not} onto: for example, \(\pi\in \R\)
        but there is no positive integer \(s\) for which \(f(s)=s/2=\pi\).
      %
\end{example}
\par

    Recall that we can combine certain functions using \emph{composition}:%
\begin{definition}[{}]\label{definition-10}
If \(f:S\to T\) and \(g:T\to U\), then the  \terminology{composition of \(f\) and \(g\)} is the function \(g\circ f: S\to U\) defined by %
\begin{equation*}
(g\circ f)(s)=g(f(s))
\end{equation*}
 for all \(s\in S\). (More generally, you can \terminology{compose} functions \(f:S\to T\) and \(g:R\to U\) to form \(g\circ f:S\to U\),  as long as \(f(S)\subseteq R\).) Also recall that given any set \(S\), the \emph{identity function on \(S\)} is the function \(1_S: S\to S\) defined by \(1_S(s)=s\) for every \(s\in S\).
  \label{notation-26}
\label{notation-27}
\end{definition}
\begin{definition}[{}]\label{definition-11}

        Let \(f\) be a function from \(S\) to \(T\). A function \(g\)
        from \(T\) to \(S\) is an \terminology{inverse} of \(f\) if \(g\circ f\) and
        \(f\circ g\) are the identity functions on \(S\) and \(T\),
        respectively; that is, if for all \(s\in S\) and \(t\in
        T\), \(g(f(s))=s\) and \(f(g(t))=t\). 

      %
\par
We say a function is \terminology{invertible} if it has an inverse.%
\end{definition}
\par
We have the following useful theorems.%
\begin{theorem}[{}]\label{invbij}

        Function \(f:S\to T\) is invertible if and only if \(f\) is a bijection. 
      %
\end{theorem}
\begin{proof}\hypertarget{proof-1}{}
Suppose \(f\) has inverse \(g\). Let \(t\in T\). Then \(f(g(t))=t\), so \(f\) is onto.  Next, suppose 
    \(s_1,s_2\in S\) with \(f(s_1)=f(s_2)\). Then %
\begin{equation*}
s_1=g(f(s_1))=g(f(s_2))=s_2,
\end{equation*}
 so \(f\)is one-to-one.%
\par
  Conversely, suppose \(f\) is a bijection.  Then for every \(t\in T\), there exists a unique element \(s_t\in S\) such that \(f(s_t)=t\). It is then straighforward to show that the function \(g:T\to S\) defined by %
\begin{equation*}
g(t)=t_s \text{ for every }t\in T
\end{equation*}
 is an inverse of \(f\).%
\end{proof}
\begin{theorem}[{}]\label{theorem-2}

     Let \(f:S\to T\) be invertible. Then the  inverse of \(f\) is unique; we denote the unique inverse of \(f\) by \(f^{-1}\).
    \label{notation-28}
\end{theorem}
\begin{proof}\hypertarget{proof-2}{}
Suppose \(f\) has inverses \(g\) and \(h\). Then for every \(t\in T,\)
    \(f(g(t))=t=f(h(t))\). But since \(f\) is invertible, it's one-to-one, so we must have \(g(t)=h(t)\) for every \(t\in T\). Thus, \(g=h\), and so \(f\) cannot have two or more distinct inverses.%
\end{proof}
\begin{theorem}[{}]\label{compbij}

        Let \(f:S\to T\) and \(g:T\to U\) be functions.
        If \(f\) and \(g\) are both 1-1 [onto], then so is \(g\circ f: S\to U\).
      %
\end{theorem}
\begin{proof}\hypertarget{proof-3}{}

      Assume \(f\) and \(g\) are 1-1. Let \(s_1, s_2\in S\)
      with \((g\circ f)(s_1)=(g\circ f)(s_2)\). Then \(g(f(s_1))=g(f(s_2))\);
      since \(g\) is one-to-one (since it's a bijection), this shows that
      \(f(s_1)=f(s_2)\). Then since \(f\) is one-to-one (since \(f\) is also a
      bijection), we must have \(s_1=s_2\). Thus, \(g\circ f\) is one-to-one.
    %
\par

    The proof that \(g\circ f\) is onto if \(f\) and \(g\) are onto is left as an exercise for the reader.
  %
\end{proof}
\typeout{************************************************}
\typeout{Section 1.3 Cardinality}
\typeout{************************************************}
\section[{Cardinality}]{Cardinality}\label{section-3}

    One of the set traits that will be useful to us in
    distinguishing between algebraic structures is \emph{cardinality}.
  %
\begin{definition}[{}]\label{definition-12}

        A set is \terminology{finite} if it contains a finite number of
        elements;
        otherwise, it's \terminology{infinite}. The \terminology{cardinality} of a finite set \(S\) is the
        number of elements in \(S\); we denote the cardinality of \(S\) by \(|S|\).
      %
\label{notation-29}
\end{definition}
\par

    \begin{warning}[]\label{warning-3}
Of course, vertical bars are used to denote
    other mathematical concepts; for instance, if \(x\) is a real
    number, \(|x|\) usually denotes the absolute value of \(x\). You
    must determine from context, and from the nature of the
    expression within the bars, what vertical bars mean in a
    particular context.%
\end{warning}

%
\par
Note that in the above definition, we omitted the definition of the cardinality of an infinite set.  This is because defining the cardinality of an infinite set is a more complicated endeavor, and one which is, in the most general context, beyond the scope of this class.  For us it will suffice to distinguish between two types of infinite sets: \emph{countably infinite} sets and \emph{uncountable} (or \emph{uncountably infinite}) sets.%
\begin{definition}[{}]\label{definition-13}
 A set \(S\) is said to be \terminology{countably infinite} if there exists a bijection from \(S\) to \(\Z\) (equivalently, if there
        exists a bijection from \(\Z\) to \(S\)). It is said to be \terminology{countable} if it is finite or countably infinite, and \terminology{uncountable} (or \terminology{uncountably infinite}) if it is not countable.%
\end{definition}
\par
  Here are some straightforward examples to get us started:%
\begin{example}[]\label{example-8}


 
    \leavevmode%
\begin{enumerate}
\item\hypertarget{li-16}{} \(|\{a,b\}|=2\). %
\item\hypertarget{li-17}{}\(|\emptyset|=0\).%
\item\hypertarget{li-18}{} Clearly, \(\Z\) itself is countably
        infinite.
      %
\end{enumerate}

  %
\end{example}
\begin{remark}[]\label{remark-3}
It turns out that countably infinite sets have the same cardinality as one another, while a countably infinite set and an uncountable set have different cardinalities. Intuitively, you  can think  of two countably infinite sets as having the same ``size,'' and a countable set and an uncountable set as having different ``sizes''; however, this is a risky way of framing things, since it can  make some results seem counterintuitive when you're used to dealing only with finite sets (see, for instance, \hyperref[zplus]{Example~\ref{zplus}}).  Luckily, exploring the cardinality of infinite sets isn't the focus of this class.%
\end{remark}
\par

    Perhaps surprisingly, a proper subset of a set can have the
    same cardinality as its superset, as the following example
    shows.
  %
\begin{example}[]\label{zplus}

        We claim that \(\Z^+\) is countably infinite. Indeed, consider the function \(f:\Z^+ \to \Z\) defined by \(f(n)=(-1)^n \lfloor n/2 \rfloor\), where \(\lfloor x \rfloor\)
        denotes the greatest integer less than or equal to \(x\), for each \(x\in \R\). The fact that \(f\) is a bijection is demonstrated (though not proven) by the following visual representation,
        where each number maps via \(f\) to the value directly to the right of it:
% group protects changes to lengths, releases boxes (?)
{% begin: group for a single side-by-side
% set panel max height to practical minimum, created in preamble
\setlength{\panelmax}{0pt}
\newsavebox{\panelboxAtabular}
\savebox{\panelboxAtabular}{%
\raisebox{\depth}{\parbox{1\textwidth}{\centering\begin{tabular}{rr}
\(1\)&\(0\)\tabularnewline[0pt]
\(1\)&\(1\)\tabularnewline[0pt]
\(2\)&\(-1\)\tabularnewline[0pt]
\(3\)&\(2\)\tabularnewline[0pt]
\(4\)&\(-2\)\tabularnewline[0pt]
\(\vdots\)&\(\vdots\)
\end{tabular}
}}}
\newlength{\phAtabular}\setlength{\phAtabular}{\ht\panelboxAtabular+\dp\panelboxAtabular}
\settototalheight{\phAtabular}{\usebox{\panelboxAtabular}}
\setlength{\panelmax}{\maxof{\panelmax}{\phAtabular}}
\leavevmode%
% begin: side-by-side as figure/tabular
% \tabcolsep change local to group
\setlength{\tabcolsep}{0\textwidth}
% @{} suppress \tabcolsep at extremes, so margins behave as intended
\begin{figure}
\begin{tabular}{@{}*{1}{c}@{}}
\begin{minipage}[c][\panelmax][t]{1\textwidth}\usebox{\panelboxAtabular}\end{minipage}\end{tabular}
\end{figure}
% end: side-by-side as tabular/figure
}% end: group for a single side-by-side

%
\par

        Note that this means that \(\Z\) and its proper subset \(\Z^+\)
        have the same cardinality, that is, the same ``size''!
      %
\end{example}
\par

    We summarize here examples of countably and uncountably
    infinite sets. (On pp. 5\textendash{}6 of~\hyperlink{F}{[1]}, Fraleigh sketches proofs of the
    facts that \(\Q\) is countable and that the interval \((0,1)\)
    in \(\R\) is uncountable. The proof then that \(\R\) is
    uncountable follows from \hyperref[cardthm]{Theorem~\ref{cardthm}}, below.)
  %
\par
Countably infinite sets: \(\Z\), \(\Z^+\), \(\Z^-\), \(\Z^*\), \(\Q\), \(\Q^+\), \(\Q^-\), \(\Q^*\), \(\N\)%
\par
Uncountably infinite sets: \(\R\), \(\R^+\), \(\R^-\), \(\R^*\), \(\C\), \(\C^*\),  the interval \((0,1)\) in \(\R\)%
\par

    The key idea here for us is that if two sets are
    essentially ``the same,'' then they must have the same ``size.''
    Thus, we see that there is some fundamental difference between
    the sets \(\Z\) and \(\R\) (in fact, there are many such
    differences). On the other hand, cardinality alone won't allow
    us to distinguish structurally between the sets \(\Z\) and
    \(\Z^+\).
  %
\par

    We end our preliminary chapter with the following theorem and a corollary of it (which can be proved using induction on \(n\)). We prove only the fourth statement in the theorem, omitting the proofs of the first two statements and leaving the proof of the third statement as an exercise for the reader.
  %
\begin{theorem}[{}]\label{cardthm}

        Let \(A\) and \(B\) be sets.
        \leavevmode%
\begin{enumerate}
\item\hypertarget{li-19}{}
              If \(A\subseteq B\) and \(A\) is infinite [uncountable] then so is \(B\).
            %
\item\hypertarget{li-20}{}
              If \(A\subseteq B\) and \(B\) is finite [countable] then so is \(A\).
            %
\item\hypertarget{li-21}{}
              If \(|A|=n\lt \infty\) and \(|B|=m\lt  \infty\), then \(|A\times B|=mn\).
            %
\item\hypertarget{li-22}{}
              Any finite product of countable sets is countable.%
\end{enumerate}

      %
\end{theorem}
\begin{proof}\hypertarget{proof-4}{}

    To prove the fourth statement: Let \(A\) and \(B\) be countable sets. Assume that \(A\) and \(B\) are both countably infinite. Since \(\Z^+\) is countably infinite, we can index the elements of \(A\) and of \(B\) by \(\Z^+\), writing
    %
\begin{equation*}
A=\{a_1,a_2,\ldots\} \text{ and }  B=\{b_1,b_2,\ldots\}.
\end{equation*}

  %
\par

    Notice that the table
%
% group protects changes to lengths, releases boxes (?)
{% begin: group for a single side-by-side
% set panel max height to practical minimum, created in preamble
\setlength{\panelmax}{0pt}
\newsavebox{\panelboxBtabular}
\savebox{\panelboxBtabular}{%
\raisebox{\depth}{\parbox{1\textwidth}{\centering\begin{tabular}{llll}
\((a_1,b_1)\)&\((a_1,b_2)\)&\((a_1,b_3)\)&\(\cdots\)\tabularnewline[0pt]
\((a_2,b_1)\)&\((a_2,b_2)\)&\((a_2,b_3)\)&\(\cdots\)\tabularnewline[0pt]
\((a_3,b_1)\)&\((a_3,b_2)\)&\((a_3,b_3)\)&\(\cdots\)\tabularnewline[0pt]
\(\vdots\)&\(\vdots\)&\(\vdots\)&\(\ddots\)
\end{tabular}
}}}
\newlength{\phBtabular}\setlength{\phBtabular}{\ht\panelboxBtabular+\dp\panelboxBtabular}
\settototalheight{\phBtabular}{\usebox{\panelboxBtabular}}
\setlength{\panelmax}{\maxof{\panelmax}{\phBtabular}}
\leavevmode%
% begin: side-by-side as figure/tabular
% \tabcolsep change local to group
\setlength{\tabcolsep}{0\textwidth}
% @{} suppress \tabcolsep at extremes, so margins behave as intended
\begin{figure}
\begin{tabular}{@{}*{1}{c}@{}}
\begin{minipage}[c][\panelmax][t]{1\textwidth}\usebox{\panelboxBtabular}\end{minipage}\end{tabular}
\end{figure}
% end: side-by-side as tabular/figure
}% end: group for a single side-by-side
\par
contains every element of \(A\times B\). We can then list the elements of \(A\times B\) by listing the elements in progressive upper-right to lower-left diagonals, starting with \((a_1,b_1)\) and moving to the right along the top row: that is, we can write
    %
\begin{equation*}
A\times B=\{(a_1,b_1),(a_1,b_2),(a_2,b_1),(a_2,b_3),(a_2,b_2),(a_3,b_1),\ldots\}.
\end{equation*}

  %
\par

    This implicitly yields a bijection from \(\Z^+\) to \(A\times B\); thus, \(A\times B\) is countably infinite, and hence countable.
  %
\par

    The proof in the case that one or both of sets \(A\) and \(B\) are finite is similar; the corresponding table in that case will simply have either only finitely many rows or finitely many columns, or both.
  %
\end{proof}
\begin{remark}[]\label{remark-4}
It is \emph{not} true that any countable product
              of countable (or even finite) sets is countable. Indeed, even
              the set \(\{0,1\}\times \{0,1\}\times \cdots\) is
              uncountable. (If you want to get into the gory details
              of this, the key is that there is a bijection from this set to
              the power set of the natural numbers, which Cantor's Theorem
              tells us is uncountable.  You are welcome to jump down this
              rabbit hole by googling ``Cantor's Theorem,'' if you desire, but
              know that you will not be responsible for that material in
              class.)%
\end{remark}
\begin{corollary}[{}]\label{corollary-1}

        Let \(n>1\) be an integer and let \(A_1,A_2,\ldots, A_n\) be countable sets. Then \(A_1\times A_2\times \cdots \times A_n\) is countable.
      %
\end{corollary}
\typeout{************************************************}
\typeout{Exercises 1.4 Exercises}
\typeout{************************************************}
\section[{Exercises}]{Exercises}\label{exercises-1}
\begin{exerciselist}
\item[1.]\hypertarget{exercise-1}{}
        Yes/No. For each of the following, write Y if the object described is a well-defined set; otherwise, write N. You do NOT need to provide explanations or show work for this problem.
        \leavevmode%
\begin{enumerate}[label=(\alph*)]
\item\hypertarget{li-23}{}
              \(\{z \in \C \,:\, |z|=1\}\)
            %
\item\hypertarget{li-24}{}
              \(\{\epsilon \in \R^+\,:\, \epsilon \mbox{ is sufficiently small} \}\)
            %
\item\hypertarget{li-25}{}
              \(\{q\in \Q \,:\, q \mbox{ can be written  with denominator } 4\}\)
            %
\item\hypertarget{li-26}{}
              \(\{n \in \Z\,:\, n^2 \lt 0\}\)
            %
\end{enumerate}

      %
\par\smallskip
\par\smallskip
\noindent\textbf{Solution.}\hypertarget{solution-1}{}\quad
\leavevmode%
\begin{multicols}{4}
\begin{enumerate}[label=(\alph*)]
\item\hypertarget{li-27}{}Y%
\item\hypertarget{li-28}{}N%
\item\hypertarget{li-29}{}Y (it's \(\Q\))%
\item\hypertarget{li-30}{}Y (it's \(\emptyset\)) %
\end{enumerate}
\end{multicols}
\item[2.]\hypertarget{exercise-2}{}
        List the elements in the following sets, writing your answers as sets.
      %
\par

        \emph{Example:} \(\{z\in \C\,:\,z^4=1\}\) \emph{Solution:} \(\{\pm 1, \pm i\}\)
        \leavevmode%
\begin{enumerate}[label=(\alph*)]
\item\hypertarget{li-31}{}
              \(\{z\in \R\,:\, z^2=5\}\)
            %
\item\hypertarget{li-32}{}
              \(\{m \in \Z\,:\, mn=50 \mbox{ for some } n\in \Z\}\)
            %
\item\hypertarget{li-33}{}
              \(\{a,b,c\}\times \{1,d\}\)
            %
\item\hypertarget{li-34}{}
              \(P(\{a,b,c\})\)
            %
\end{enumerate}

      %
\par\smallskip
\par\smallskip
\noindent\textbf{Solution.}\hypertarget{solution-2}{}\quad
\leavevmode%
\begin{enumerate}[label=(\alph*)]
\item\hypertarget{li-35}{}
          \(\{\pm\sqrt{5}\}\)
        %
\item\hypertarget{li-36}{}
          \(\{\pm 50, \pm 25, \pm 10, \pm 5, \pm 2, \pm 1\}\)
        %
\item\hypertarget{li-37}{}
          \(\{(a,1),(a,d), (b,1),(b,d),(c,1),(c,d)\}\)
        %
\item\hypertarget{li-38}{}
          \(\{\emptyset, \{a\}, \{b\},
          \{c\},\{a,b\},\{a,c\},\{b,c\}, \{a,b,c\}\}\)
        %
\end{enumerate}
\item[3.]\hypertarget{exercise-3}{}
        Let \(S\) be a set with cardinality \(n\in \N\). Use the cardinalities of \(P(\{a,b\})\) and \(P(\{a,b,c\})\) to make a conjecture about the cardinality of \(P(S)\). You do not need to prove that your conjecture is correct (but you should try to ensure it is correct).
      %
\par\smallskip
\par\smallskip
\noindent\textbf{Solution.}\hypertarget{solution-3}{}\quad

      \(|P(\{a,b\})|=4=2^2\) and \(|P(\{a,b,c\})|=8=2^3\); we may conjecture that when \(|S|=n\), \(|P(S)|=2^n\).
    %
\item[4.]\hypertarget{exercise-4}{}
        Let \(f: \Z^2 \to \R\) be defined by \(f(a,b)=ab\). (Note: technically, we should write \(f((a,b))\), not \(f(a,b)\), since \(f\) is being applied to an ordered pair, but this is one of those cases in which mathematicians abuse notation in the interest of concision.)
      %
\leavevmode%
\begin{enumerate}[label=(\alph*)]
\item\hypertarget{li-39}{}
          What are \(f\)'s domain, codomain, and range?
        %
\item\hypertarget{li-40}{}
          Prove or disprove each of the following statements. (Your proofs do not need to be long to be correct!)
        %
%
\begin{enumerate}[label=\roman*.]
\item\hypertarget{li-41}{}
              \(f\) is onto;
            %
\item\hypertarget{li-42}{}
              \(f\) is 1-1;
            %
\item\hypertarget{li-43}{}
              \(f\) is a bijection. (You may refer to parts (i) and (ii) for this part.)
            %
\end{enumerate}
\item\hypertarget{li-44}{}
          Find the images of the element \((6,-2)\) and of the set \(\Z^- \times \Z^-\) under \(f\). (Remember that the image of an element is an element, and the image of a set is a set.)
        %
\item\hypertarget{li-45}{}
          Find the preimage of \(\{2,3\}\) under \(f\). (Remember that the preimage of a set is a set.)
        %
\end{enumerate}
\par\smallskip
\par\smallskip
\noindent\textbf{Solution.}\hypertarget{solution-4}{}\quad
\leavevmode%
\begin{enumerate}[label=(\alph*)]
\item\hypertarget{li-46}{}
        \(f\)'s domain, codomain, and range are, respectively, \(\Z^2\), \(\R\), and \(\Z\).
      %
\item\hypertarget{li-47}{}%
\begin{enumerate}[label=\roman*.]
\item\hypertarget{li-48}{}
            \(f\) is not onto, since, for instance, \(1/2\in \R-\Z\).
          %
\item\hypertarget{li-49}{}
            \(f\) is not 1-1: for instance, \(f(-2,2)=-4=f(2,-2)\).
          %
\item\hypertarget{li-50}{}
            \(f\) is not a bijection since it's not 1-1. (It would be equally valid to answer that it's not a bijection since it's not onto, or that it's not a bijection since it's neither 1-1 nor onto.)
          %
\end{enumerate}
%
\item\hypertarget{li-51}{}
        \(f(6,-2)=-12\) and \(f(\Z^-\times \Z^-)=\Z^+\).
      %
\item\hypertarget{li-52}{}
        %
\begin{align*}
f^{\leftarrow}(\{2,3\})\amp =\{(a,b)\in \Z\times \Z\,:\, f(a,b)\in \{2,3\}\}\\
\amp =\{(a,b)\in \Z\times \Z\,:\, ab=2 \mbox{ or }  ab=3\},
\end{align*}

        which is the set
        %
\begin{equation*}
\{(1,2),(2,1),(-1,-2),(-2,-1),(1,3),(3,1),(-1,-3),(-3,-1)\}.
\end{equation*}

      %
\end{enumerate}
\item[5.]\hypertarget{exercise-5}{}
        Let \(S\), \(T\), and \(U\) be sets, and let \(f: S\to T\) and \(g: T\to U\) be onto. Prove that \(g \circ f\) is onto.
      %
\par\smallskip
\par\smallskip
\noindent\textbf{Solution.}\hypertarget{solution-5}{}\quad

      Let \(u\in U\). We want to show there is an element of \(S\) that gets mapped to \(u\) by \(g\circ f\). Since \(g:T\to U\) is onto, there is an element \(t\in T\) such that \(g(t)=u\); next, since \(f:S\to T\) is onto, there is an element \(s\in S\) such that \(f(s)=t\). Then \((g\circ f)(s)=g(f(s))=g(t)=u\). Thus, \(g\circ f\) is onto.
    %
\item[6.]\hypertarget{exercise-6}{}
        Let \(|A|=n\lt \infty\) and \(|B|=m\lt  \infty\). Prove that \(|A\times B|=mn\).
      %
\par\smallskip
\par\smallskip
\noindent\textbf{Solution.}\hypertarget{solution-6}{}\quad

      We can list the elements of \(A\) and \(B\) as so:
      %
\begin{equation*}
 A=\{a_1,a_2,\ldots, a_n\} \mbox{ and } B=\{b_1,b_2,\ldots, b_m\}.
\end{equation*}

    %
\par

      Consider the table
  %
\leavevmode%
\begin{table}
\centering
\begin{tabular}{cccc}
\((a_1,b_1)\)&\((a_1,b_2)\)&\(\cdots\)&\((a_1,b_m)\)\tabularnewline[0pt]
\((a_2,b_1)\)&\((a_2,b_2)\)&\(\cdots\)&\((a_2,b_m)\)\tabularnewline[0pt]
\(\vdots\)&\(\vdots\)&\(\ddots\)&\(\vdots\)\tabularnewline[0pt]
\((a_n,b_1)\)&\((a_n,b_2)\)&\(\cdots\)&\((a_n,b_m)\)
\end{tabular}
\caption{Elements of \(A\times B\) when \(|A|=n\) and \(|B|=m\)\label{cardmn}}
\end{table}
\par

      Clearly this table contains \(mn\) elements, and contains each element of \(A\times B\) exactly once. Therefore, \(|A\times B|=mn\).
    %
\end{exerciselist}
%
%% A lineskip in table of contents as transition to appendices, backmatter
\addtocontents{toc}{\vspace{\normalbaselineskip}}
%
%
\appendix
%
\typeout{************************************************}
\typeout{Appendix A Notation}
\typeout{************************************************}
\chapter[{Notation}]{Notation}\label{appendix-1}
The following table defines the notation used in this book. Page numbers or references refer to the first appearance of each symbol.%
\begin{longtable}[l]{lp{0.60\textwidth}r}
\textbf{Symbol}&\textbf{Description}&\textbf{Page}\\[1em]
\endfirsthead
\textbf{Symbol}&\textbf{Description}&\textbf{Page}\\[1em]
\endhead
\multicolumn{3}{r}{(Continued on next page)}\\
\endfoot
\endlastfoot
$x \in S$&\(x\) is an element of \(S\)&\pageref{notation-1}\\
$x \not\in S$&\(x\) is not an element of \(S\)&\pageref{notation-2}\\
$\emptyset$&the empty set, \(\{\}\)&\pageref{notation-3}\\
$\Z$&the set of all integers&\pageref{notation-4}\\
$\Q$&the set of all rational numbers&\pageref{notation-5}\\
$\R$&the set of all real numbers&\pageref{notation-6}\\
$\C$&the set of all complex numbers&\pageref{notation-7}\\
$\N$&the set of all natural numbers, \(\{0,1,2,\ldots\}\)&\pageref{notation-8}\\
$\Z^+,\Q^+,\R^+$&the set of all positive elements of \(\Z,\Q,\R\)&\pageref{notation-9}\\
$\Z^-,\Q^-,\R^-$&the set of all negative elements of \(\Z,\Q,\R\)&\pageref{notation-10}\\
$\Z^*,\Q^*,\R^*,\C^*$&the set of all nonzero elements of \(\Z,\Q,\R,\C\)&\pageref{notation-11}\\
$\M_{m\times n}(S)$&the set of all \(m \times n\) matrices over \(S\)&\pageref{notation-12}\\
$\M_n(S)$&the set of all \(n \times n\) matrices over \(S\)&\pageref{notation-13}\\
$A\subseteq B$& \(A\) is a subset of the \(B\)&\pageref{notation-14}\\
$A\subsetneq B$& \(A\) is a proper subset of \(B\)&\pageref{notation-15}\\
$P(A)$&the power set of \(A\)&\pageref{notation-16}\\
$A\cap B$&the intersection of  \(A\) and \(B\)&\pageref{notation-17}\\
$A\cup B$&the union of  \(A\) and \(B\)&\pageref{notation-18}\\
$A - B$&the difference of  \(A\) and \(B\)&\pageref{notation-19}\\
$\bigcup_{i\in I}A_i$&\(\{x: x\in A_i \text{ for some }  i\in I\}\)&\pageref{notation-20}\\
$\bigcap_{i\in I}A_i$&\(\{x: x\in A_i \text{ for every }  i\in I\}\)&\pageref{notation-21}\\
$A\times B$&the direct product of  \(A\) and \(B\)&\pageref{notation-22}\\
$f:S\to T$&function \(f\) from  \(S\) to  \(T\)&\pageref{notation-23}\\
$f(U)$&the image of a set \(U\) under \(f\)&\pageref{notation-24}\\
$f^{\leftarrow}(V)$&the preimage of a set \(V\) under \(f\)&\pageref{notation-25}\\
$f\circ g$&the composition of  \(f\) with  \(g\)&\pageref{notation-26}\\
$1_S$&the identity function on \(S\)&\pageref{notation-27}\\
$f^{-1}$&the inverse of \(f\)&\pageref{notation-28}\\
$|S|$&the cardinality of  \(S\)&\pageref{notation-29}\\
\end{longtable}
\typeout{************************************************}
\typeout{Appendix B Solutions to Exercises}
\typeout{************************************************}
\chapter[{Solutions to Exercises}]{Solutions to Exercises}\label{appendix-2}
\section*{1.4 Exercises}
\noindent\textbf{1.}\quad{}
        Yes/No. For each of the following, write Y if the object described is a well-defined set; otherwise, write N. You do NOT need to provide explanations or show work for this problem.
        \leavevmode%
\begin{enumerate}[label=(\alph*)]
\item\hypertarget{li-23}{}
              \(\{z \in \C \,:\, |z|=1\}\)
            %
\item\hypertarget{li-24}{}
              \(\{\epsilon \in \R^+\,:\, \epsilon \mbox{ is sufficiently small} \}\)
            %
\item\hypertarget{li-25}{}
              \(\{q\in \Q \,:\, q \mbox{ can be written  with denominator } 4\}\)
            %
\item\hypertarget{li-26}{}
              \(\{n \in \Z\,:\, n^2 \lt 0\}\)
            %
\end{enumerate}

      %
\par\smallskip
\leavevmode%
\begin{multicols}{4}
\begin{enumerate}[label=(\alph*)]
\item\hypertarget{li-27}{}Y%
\item\hypertarget{li-28}{}N%
\item\hypertarget{li-29}{}Y (it's \(\Q\))%
\item\hypertarget{li-30}{}Y (it's \(\emptyset\)) %
\end{enumerate}
\end{multicols}
\par\smallskip
\noindent\textbf{2.}\quad{}
        List the elements in the following sets, writing your answers as sets.
      %
\par

        \emph{Example:} \(\{z\in \C\,:\,z^4=1\}\) \emph{Solution:} \(\{\pm 1, \pm i\}\)
        \leavevmode%
\begin{enumerate}[label=(\alph*)]
\item\hypertarget{li-31}{}
              \(\{z\in \R\,:\, z^2=5\}\)
            %
\item\hypertarget{li-32}{}
              \(\{m \in \Z\,:\, mn=50 \mbox{ for some } n\in \Z\}\)
            %
\item\hypertarget{li-33}{}
              \(\{a,b,c\}\times \{1,d\}\)
            %
\item\hypertarget{li-34}{}
              \(P(\{a,b,c\})\)
            %
\end{enumerate}

      %
\par\smallskip
\leavevmode%
\begin{enumerate}[label=(\alph*)]
\item\hypertarget{li-35}{}
          \(\{\pm\sqrt{5}\}\)
        %
\item\hypertarget{li-36}{}
          \(\{\pm 50, \pm 25, \pm 10, \pm 5, \pm 2, \pm 1\}\)
        %
\item\hypertarget{li-37}{}
          \(\{(a,1),(a,d), (b,1),(b,d),(c,1),(c,d)\}\)
        %
\item\hypertarget{li-38}{}
          \(\{\emptyset, \{a\}, \{b\},
          \{c\},\{a,b\},\{a,c\},\{b,c\}, \{a,b,c\}\}\)
        %
\end{enumerate}
\par\smallskip
\noindent\textbf{3.}\quad{}
        Let \(S\) be a set with cardinality \(n\in \N\). Use the cardinalities of \(P(\{a,b\})\) and \(P(\{a,b,c\})\) to make a conjecture about the cardinality of \(P(S)\). You do not need to prove that your conjecture is correct (but you should try to ensure it is correct).
      %
\par\smallskip

      \(|P(\{a,b\})|=4=2^2\) and \(|P(\{a,b,c\})|=8=2^3\); we may conjecture that when \(|S|=n\), \(|P(S)|=2^n\).
    %
\par\smallskip
\noindent\textbf{4.}\quad{}
        Let \(f: \Z^2 \to \R\) be defined by \(f(a,b)=ab\). (Note: technically, we should write \(f((a,b))\), not \(f(a,b)\), since \(f\) is being applied to an ordered pair, but this is one of those cases in which mathematicians abuse notation in the interest of concision.)
      %
\leavevmode%
\begin{enumerate}[label=(\alph*)]
\item\hypertarget{li-39}{}
          What are \(f\)'s domain, codomain, and range?
        %
\item\hypertarget{li-40}{}
          Prove or disprove each of the following statements. (Your proofs do not need to be long to be correct!)
        %
%
\begin{enumerate}[label=\roman*.]
\item\hypertarget{li-41}{}
              \(f\) is onto;
            %
\item\hypertarget{li-42}{}
              \(f\) is 1-1;
            %
\item\hypertarget{li-43}{}
              \(f\) is a bijection. (You may refer to parts (i) and (ii) for this part.)
            %
\end{enumerate}
\item\hypertarget{li-44}{}
          Find the images of the element \((6,-2)\) and of the set \(\Z^- \times \Z^-\) under \(f\). (Remember that the image of an element is an element, and the image of a set is a set.)
        %
\item\hypertarget{li-45}{}
          Find the preimage of \(\{2,3\}\) under \(f\). (Remember that the preimage of a set is a set.)
        %
\end{enumerate}
\par\smallskip
\leavevmode%
\begin{enumerate}[label=(\alph*)]
\item\hypertarget{li-46}{}
        \(f\)'s domain, codomain, and range are, respectively, \(\Z^2\), \(\R\), and \(\Z\).
      %
\item\hypertarget{li-47}{}%
\begin{enumerate}[label=\roman*.]
\item\hypertarget{li-48}{}
            \(f\) is not onto, since, for instance, \(1/2\in \R-\Z\).
          %
\item\hypertarget{li-49}{}
            \(f\) is not 1-1: for instance, \(f(-2,2)=-4=f(2,-2)\).
          %
\item\hypertarget{li-50}{}
            \(f\) is not a bijection since it's not 1-1. (It would be equally valid to answer that it's not a bijection since it's not onto, or that it's not a bijection since it's neither 1-1 nor onto.)
          %
\end{enumerate}
%
\item\hypertarget{li-51}{}
        \(f(6,-2)=-12\) and \(f(\Z^-\times \Z^-)=\Z^+\).
      %
\item\hypertarget{li-52}{}
        %
\begin{align*}
f^{\leftarrow}(\{2,3\})\amp =\{(a,b)\in \Z\times \Z\,:\, f(a,b)\in \{2,3\}\}\\
\amp =\{(a,b)\in \Z\times \Z\,:\, ab=2 \mbox{ or }  ab=3\},
\end{align*}

        which is the set
        %
\begin{equation*}
\{(1,2),(2,1),(-1,-2),(-2,-1),(1,3),(3,1),(-1,-3),(-3,-1)\}.
\end{equation*}

      %
\end{enumerate}
\par\smallskip
\noindent\textbf{5.}\quad{}
        Let \(S\), \(T\), and \(U\) be sets, and let \(f: S\to T\) and \(g: T\to U\) be onto. Prove that \(g \circ f\) is onto.
      %
\par\smallskip

      Let \(u\in U\). We want to show there is an element of \(S\) that gets mapped to \(u\) by \(g\circ f\). Since \(g:T\to U\) is onto, there is an element \(t\in T\) such that \(g(t)=u\); next, since \(f:S\to T\) is onto, there is an element \(s\in S\) such that \(f(s)=t\). Then \((g\circ f)(s)=g(f(s))=g(t)=u\). Thus, \(g\circ f\) is onto.
    %
\par\smallskip
\noindent\textbf{6.}\quad{}
        Let \(|A|=n\lt \infty\) and \(|B|=m\lt  \infty\). Prove that \(|A\times B|=mn\).
      %
\par\smallskip

      We can list the elements of \(A\) and \(B\) as so:
      %
\begin{equation*}
 A=\{a_1,a_2,\ldots, a_n\} \mbox{ and } B=\{b_1,b_2,\ldots, b_m\}.
\end{equation*}

    %
\par

      Consider the table
  %
\leavevmode%
\begin{table}
\centering
\begin{tabular}{cccc}
\((a_1,b_1)\)&\((a_1,b_2)\)&\(\cdots\)&\((a_1,b_m)\)\tabularnewline[0pt]
\((a_2,b_1)\)&\((a_2,b_2)\)&\(\cdots\)&\((a_2,b_m)\)\tabularnewline[0pt]
\(\vdots\)&\(\vdots\)&\(\ddots\)&\(\vdots\)\tabularnewline[0pt]
\((a_n,b_1)\)&\((a_n,b_2)\)&\(\cdots\)&\((a_n,b_m)\)
\end{tabular}
\caption{Elements of \(A\times B\) when \(|A|=n\) and \(|B|=m\)\label{cardmn}}
\end{table}
\par

      Clearly this table contains \(mn\) elements, and contains each element of \(A\times B\) exactly once. Therefore, \(|A\times B|=mn\).
    %
\par\smallskip
\typeout{************************************************}
\typeout{References  Bibliography}
\typeout{************************************************}
\chapter[{Bibliography}]{Bibliography}\label{references-1}
%% If this is a top-level references
%%   you can replace with "thebibliography" environment
\begin{referencelist}
\bibitem[1]{F}\hypertarget{F}{}John B. Fraleigh, \textit{A First Course in Abstract Algebra} (7th ed.), Addison Wesley, 2002.
\bibitem[2]{J}\hypertarget{J}{}Thomas W. Judson, \textit{Abstract Algebra: Theory and Applications}. Revised edition published under the GNU Free
Documentation License, 1997 (revised 2016). 
\href{http://abstract.pugetsound.edu/aata}{http://abstract.pugetsound.edu/aata}

\bibitem[3]{NZM}\hypertarget{NZM}{}Ivan Niven, Herbert S. Zuckerman, and Hugh L.
Montgomery. \textit{An Introduction to the Theory of Numbers} (5th
ed.), John Wiley and Sons, 1991.


\end{referencelist}
\typeout{************************************************}
\typeout{Appendix C GNU Free Documentation License}
\typeout{************************************************}
\chapter[{GNU Free Documentation License}]{GNU Free Documentation License}\label{appendix-gfdl}
\typeout{************************************************}
\typeout{Paragraphs  }
\typeout{************************************************}
\paragraph[{}]{}\hypertarget{paragraphs-1}{}
Version 1.3, 3 November 2008%
\par
Copyright \textcopyright{} 2000, 2001, 2002, 2007, 2008 Free Software Foundation, Inc. \textless{}\url{http://www.fsf.org/}\textgreater{}%
\par
Everyone is permitted to copy and distribute verbatim copies of this license document, but changing it is not allowed.%
\typeout{************************************************}
\typeout{Paragraphs  0. PREAMBLE}
\typeout{************************************************}
\paragraph[{0. PREAMBLE}]{0. PREAMBLE}\hypertarget{gfdl-section0}{}
The purpose of this License is to make a manual, textbook, or other functional and useful document ``free'' in the sense of freedom: to assure everyone the effective freedom to copy and redistribute it, with or without modifying it, either commercially or noncommercially. Secondarily, this License preserves for the author and publisher a way to get credit for their work, while not being considered responsible for modifications made by others.%
\par
This License is a kind of ``copyleft'', which means that derivative works of the document must themselves be free in the same sense. It complements the GNU General Public License, which is a copyleft license designed for free software.%
\par
We have designed this License in order to use it for manuals for free software, because free software needs free documentation: a free program should come with manuals providing the same freedoms that the software does. But this License is not limited to software manuals; it can be used for any textual work, regardless of subject matter or whether it is published as a printed book. We recommend this License principally for works whose purpose is instruction or reference.%
\typeout{************************************************}
\typeout{Paragraphs  1. APPLICABILITY AND DEFINITIONS}
\typeout{************************************************}
\paragraph[{1. APPLICABILITY AND DEFINITIONS}]{1. APPLICABILITY AND DEFINITIONS}\hypertarget{gfdl-section1}{}
This License applies to any manual or other work, in any medium, that contains a notice placed by the copyright holder saying it can be distributed under the terms of this License. Such a notice grants a world-wide, royalty-free license, unlimited in duration, to use that work under the conditions stated herein. The ``Document'', below, refers to any such manual or work. Any member of the public is a licensee, and is addressed as ``you''. You accept the license if you copy, modify or distribute the work in a way requiring permission under copyright law.%
\par
A ``Modified Version'' of the Document means any work containing the Document or a portion of it, either copied verbatim, or with modifications and/or translated into another language.%
\par
A ``Secondary Section'' is a named appendix or a front-matter section of the Document that deals exclusively with the relationship of the publishers or authors of the Document to the Document's overall subject (or to related matters) and contains nothing that could fall directly within that overall subject. (Thus, if the Document is in part a textbook of mathematics, a Secondary Section may not explain any mathematics.) The relationship could be a matter of historical connection with the subject or with related matters, or of legal, commercial, philosophical, ethical or political position regarding them.%
\par
The ``Invariant Sections'' are certain Secondary Sections whose titles are designated, as being those of Invariant Sections, in the notice that says that the Document is released under this License. If a section does not fit the above definition of Secondary then it is not allowed to be designated as Invariant. The Document may contain zero Invariant Sections. If the Document does not identify any Invariant Sections then there are none.%
\par
The ``Cover Texts'' are certain short passages of text that are listed, as Front-Cover Texts or Back-Cover Texts, in the notice that says that the Document is released under this License. A Front-Cover Text may be at most 5 words, and a Back-Cover Text may be at most 25 words.%
\par
A ``Transparent'' copy of the Document means a machine-readable copy, represented in a format whose specification is available to the general public, that is suitable for revising the document straightforwardly with generic text editors or (for images composed of pixels) generic paint programs or (for drawings) some widely available drawing editor, and that is suitable for input to text formatters or for automatic translation to a variety of formats suitable for input to text formatters. A copy made in an otherwise Transparent file format whose markup, or absence of markup, has been arranged to thwart or discourage subsequent modification by readers is not Transparent. An image format is not Transparent if used for any substantial amount of text. A copy that is not ``Transparent'' is called ``Opaque''.%
\par
Examples of suitable formats for Transparent copies include plain ASCII without markup, Texinfo input format, LaTeX input format, SGML or XML using a publicly available DTD, and standard-conforming simple HTML, PostScript or PDF designed for human modification. Examples of transparent image formats include PNG, XCF and JPG. Opaque formats include proprietary formats that can be read and edited only by proprietary word processors, SGML or XML for which the DTD and/or processing tools are not generally available, and the machine-generated HTML, PostScript or PDF produced by some word processors for output purposes only.%
\par
The ``Title Page'' means, for a printed book, the title page itself, plus such following pages as are needed to hold, legibly, the material this License requires to appear in the title page. For works in formats which do not have any title page as such, ``Title Page'' means the text near the most prominent appearance of the work's title, preceding the beginning of the body of the text.%
\par
The ``publisher'' means any person or entity that distributes copies of the Document to the public.%
\par
A section ``Entitled XYZ'' means a named subunit of the Document whose title either is precisely XYZ or contains XYZ in parentheses following text that translates XYZ in another language. (Here XYZ stands for a specific section name mentioned below, such as ``Acknowledgements'', ``Dedications'', ``Endorsements'', or ``History''.) To ``Preserve the Title'' of such a section when you modify the Document means that it remains a section ``Entitled XYZ'' according to this definition.%
\par
The Document may include Warranty Disclaimers next to the notice which states that this License applies to the Document. These Warranty Disclaimers are considered to be included by reference in this License, but only as regards disclaiming warranties: any other implication that these Warranty Disclaimers may have is void and has no effect on the meaning of this License.%
\typeout{************************************************}
\typeout{Paragraphs  2. VERBATIM COPYING}
\typeout{************************************************}
\paragraph[{2. VERBATIM COPYING}]{2. VERBATIM COPYING}\hypertarget{gfdl-section2}{}
You may copy and distribute the Document in any medium, either commercially or noncommercially, provided that this License, the copyright notices, and the license notice saying this License applies to the Document are reproduced in all copies, and that you add no other conditions whatsoever to those of this License. You may not use technical measures to obstruct or control the reading or further copying of the copies you make or distribute. However, you may accept compensation in exchange for copies. If you distribute a large enough number of copies you must also follow the conditions in section 3.%
\par
You may also lend copies, under the same conditions stated above, and you may publicly display copies.%
\typeout{************************************************}
\typeout{Paragraphs  3. COPYING IN QUANTITY}
\typeout{************************************************}
\paragraph[{3. COPYING IN QUANTITY}]{3. COPYING IN QUANTITY}\hypertarget{gfdl-section3}{}
If you publish printed copies (or copies in media that commonly have printed covers) of the Document, numbering more than 100, and the Document's license notice requires Cover Texts, you must enclose the copies in covers that carry, clearly and legibly, all these Cover Texts: Front-Cover Texts on the front cover, and Back-Cover Texts on the back cover. Both covers must also clearly and legibly identify you as the publisher of these copies. The front cover must present the full title with all words of the title equally prominent and visible. You may add other material on the covers in addition. Copying with changes limited to the covers, as long as they preserve the title of the Document and satisfy these conditions, can be treated as verbatim copying in other respects.%
\par
If the required texts for either cover are too voluminous to fit legibly, you should put the first ones listed (as many as fit reasonably) on the actual cover, and continue the rest onto adjacent pages.%
\par
If you publish or distribute Opaque copies of the Document numbering more than 100, you must either include a machine-readable Transparent copy along with each Opaque copy, or state in or with each Opaque copy a computer-network location from which the general network-using public has access to download using public-standard network protocols a complete Transparent copy of the Document, free of added material. If you use the latter option, you must take reasonably prudent steps, when you begin distribution of Opaque copies in quantity, to ensure that this Transparent copy will remain thus accessible at the stated location until at least one year after the last time you distribute an Opaque copy (directly or through your agents or retailers) of that edition to the public.%
\par
It is requested, but not required, that you contact the authors of the Document well before redistributing any large number of copies, to give them a chance to provide you with an updated version of the Document.%
\typeout{************************************************}
\typeout{Paragraphs  4. MODIFICATIONS}
\typeout{************************************************}
\paragraph[{4. MODIFICATIONS}]{4. MODIFICATIONS}\hypertarget{gfdl-section4}{}
You may copy and distribute a Modified Version of the Document under the conditions of sections 2 and 3 above, provided that you release the Modified Version under precisely this License, with the Modified Version filling the role of the Document, thus licensing distribution and modification of the Modified Version to whoever possesses a copy of it. In addition, you must do these things in the Modified Version:%
\leavevmode%
\begin{enumerate}[label=\Alph*.]
\item\hypertarget{li-53}{}Use in the Title Page (and on the covers, if any) a title distinct from that of the Document, and from those of previous versions (which should, if there were any, be listed in the History section of the Document). You may use the same title as a previous version if the original publisher of that version gives permission.%
\item\hypertarget{li-54}{}List on the Title Page, as authors, one or more persons or entities responsible for authorship of the modifications in the Modified Version, together with at least five of the principal authors of the Document (all of its principal authors, if it has fewer than five), unless they release you from this requirement.%
\item\hypertarget{li-55}{}State on the Title page the name of the publisher of the Modified Version, as the publisher.%
\item\hypertarget{li-56}{}Preserve all the copyright notices of the Document.%
\item\hypertarget{li-57}{}Add an appropriate copyright notice for your modifications adjacent to the other copyright notices.%
\item\hypertarget{li-58}{}Include, immediately after the copyright notices, a license notice giving the public permission to use the Modified Version under the terms of this License, in the form shown in the Addendum below.%
\item\hypertarget{li-59}{}Preserve in that license notice the full lists of Invariant Sections and required Cover Texts given in the Document's license notice.%
\item\hypertarget{li-60}{}Include an unaltered copy of this License.%
\item\hypertarget{li-61}{}Preserve the section Entitled ``History'', Preserve its Title, and add to it an item stating at least the title, year, new authors, and publisher of the Modified Version as given on the Title Page. If there is no section Entitled ``History'' in the Document, create one stating the title, year, authors, and publisher of the Document as given on its Title Page, then add an item describing the Modified Version as stated in the previous sentence.%
\item\hypertarget{li-62}{}Preserve the network location, if any, given in the Document for public access to a Transparent copy of the Document, and likewise the network locations given in the Document for previous versions it was based on.  These may be placed in the ``History'' section. You may omit a network location for a work that was published at least four years before the Document itself, or if the original publisher of the version it refers to gives permission.%
\item\hypertarget{li-63}{}For any section Entitled ``Acknowledgements'' or ``Dedications'', Preserve the Title of the section, and preserve in the section all the substance and tone of each of the contributor acknowledgements and/or dedications given therein.%
\item\hypertarget{li-64}{}Preserve all the Invariant Sections of the Document, unaltered in their text and in their titles. Section numbers or the equivalent are not considered part of the section titles.%
\item\hypertarget{li-65}{}Delete any section Entitled ``Endorsements''. Such a section may not be included in the Modified Version.%
\item\hypertarget{li-66}{}Do not retitle any existing section to be Entitled ``Endorsements'' or to conflict in title with any Invariant Section.%
\item\hypertarget{li-67}{}Preserve any Warranty Disclaimers.%
\end{enumerate}
\par
If the Modified Version includes new front-matter sections or appendices that qualify as Secondary Sections and contain no material copied from the Document, you may at your option designate some or all of these sections as invariant. To do this, add their titles to the list of Invariant Sections in the Modified Version's license notice. These titles must be distinct from any other section titles.%
\par
You may add a section Entitled ``Endorsements'', provided it contains nothing but endorsements of your Modified Version by various parties \textemdash{} for example, statements of peer review or that the text has been approved by an organization as the authoritative definition of a standard.%
\par
You may add a passage of up to five words as a Front-Cover Text, and a passage of up to 25 words as a Back-Cover Text, to the end of the list of Cover Texts in the Modified Version. Only one passage of Front-Cover Text and one of Back-Cover Text may be added by (or through arrangements made by) any one entity. If the Document already includes a cover text for the same cover, previously added by you or by arrangement made by the same entity you are acting on behalf of, you may not add another; but you may replace the old one, on explicit permission from the previous publisher that added the old one.%
\par
The author(s) and publisher(s) of the Document do not by this License give permission to use their names for publicity for or to assert or imply endorsement of any Modified Version.%
\typeout{************************************************}
\typeout{Paragraphs  5. COMBINING DOCUMENTS}
\typeout{************************************************}
\paragraph[{5. COMBINING DOCUMENTS}]{5. COMBINING DOCUMENTS}\hypertarget{gfdl-section5}{}
You may combine the Document with other documents released under this License, under the terms defined in section 4 above for modified versions, provided that you include in the combination all of the Invariant Sections of all of the original documents, unmodified, and list them all as Invariant Sections of your combined work in its license notice, and that you preserve all their Warranty Disclaimers.%
\par
The combined work need only contain one copy of this License, and multiple identical Invariant Sections may be replaced with a single copy. If there are multiple Invariant Sections with the same name but different contents, make the title of each such section unique by adding at the end of it, in parentheses, the name of the original author or publisher of that section if known, or else a unique number. Make the same adjustment to the section titles in the list of Invariant Sections in the license notice of the combined work.%
\par
In the combination, you must combine any sections Entitled ``History'' in the various original documents, forming one section Entitled ``History''; likewise combine any sections Entitled ``Acknowledgements'', and any sections Entitled ``Dedications''. You must delete all sections Entitled ``Endorsements''.%
\typeout{************************************************}
\typeout{Paragraphs  6. COLLECTIONS OF DOCUMENTS}
\typeout{************************************************}
\paragraph[{6. COLLECTIONS OF DOCUMENTS}]{6. COLLECTIONS OF DOCUMENTS}\hypertarget{gfdl-section6}{}
You may make a collection consisting of the Document and other documents released under this License, and replace the individual copies of this License in the various documents with a single copy that is included in the collection, provided that you follow the rules of this License for verbatim copying of each of the documents in all other respects.%
\par
You may extract a single document from such a collection, and distribute it individually under this License, provided you insert a copy of this License into the extracted document, and follow this License in all other respects regarding verbatim copying of that document.%
\typeout{************************************************}
\typeout{Paragraphs  7. AGGREGATION WITH INDEPENDENT WORKS}
\typeout{************************************************}
\paragraph[{7. AGGREGATION WITH INDEPENDENT WORKS}]{7. AGGREGATION WITH INDEPENDENT WORKS}\hypertarget{gfdl-section7}{}
A compilation of the Document or its derivatives with other separate and independent documents or works, in or on a volume of a storage or distribution medium, is called an ``aggregate'' if the copyright resulting from the compilation is not used to limit the legal rights of the compilation's users beyond what the individual works permit. When the Document is included in an aggregate, this License does not apply to the other works in the aggregate which are not themselves derivative works of the Document.%
\par
If the Cover Text requirement of section 3 is applicable to these copies of the Document, then if the Document is less than one half of the entire aggregate, the Document's Cover Texts may be placed on covers that bracket the Document within the aggregate, or the electronic equivalent of covers if the Document is in electronic form. Otherwise they must appear on printed covers that bracket the whole aggregate.%
\typeout{************************************************}
\typeout{Paragraphs  8. TRANSLATION}
\typeout{************************************************}
\paragraph[{8. TRANSLATION}]{8. TRANSLATION}\hypertarget{gfdl-section8}{}
Translation is considered a kind of modification, so you may distribute translations of the Document under the terms of section 4. Replacing Invariant Sections with translations requires special permission from their copyright holders, but you may include translations of some or all Invariant Sections in addition to the original versions of these Invariant Sections. You may include a translation of this License, and all the license notices in the Document, and any Warranty Disclaimers, provided that you also include the original English version of this License and the original versions of those notices and disclaimers. In case of a disagreement between the translation and the original version of this License or a notice or disclaimer, the original version will prevail.%
\par
If a section in the Document is Entitled ``Acknowledgements'', ``Dedications'', or ``History'', the requirement (section 4) to Preserve its Title (section 1) will typically require changing the actual title.%
\typeout{************************************************}
\typeout{Paragraphs  9. TERMINATION}
\typeout{************************************************}
\paragraph[{9. TERMINATION}]{9. TERMINATION}\hypertarget{gfdl-section9}{}
You may not copy, modify, sublicense, or distribute the Document except as expressly provided under this License. Any attempt otherwise to copy, modify, sublicense, or distribute it is void, and will automatically terminate your rights under this License.%
\par
However, if you cease all violation of this License, then your license from a particular copyright holder is reinstated (a) provisionally, unless and until the copyright holder explicitly and finally terminates your license, and (b) permanently, if the copyright holder fails to notify you of the violation by some reasonable means prior to 60 days after the cessation.%
\par
Moreover, your license from a particular copyright holder is reinstated permanently if the copyright holder notifies you of the violation by some reasonable means, this is the first time you have received notice of violation of this License (for any work) from that copyright holder, and you cure the violation prior to 30 days after your receipt of the notice.%
\par
Termination of your rights under this section does not terminate the licenses of parties who have received copies or rights from you under this License. If your rights have been terminated and not permanently reinstated, receipt of a copy of some or all of the same material does not give you any rights to use it.%
\typeout{************************************************}
\typeout{Paragraphs  10. FUTURE REVISIONS OF THIS LICENSE}
\typeout{************************************************}
\paragraph[{10. FUTURE REVISIONS OF THIS LICENSE}]{10. FUTURE REVISIONS OF THIS LICENSE}\hypertarget{gfdl-section10}{}
The Free Software Foundation may publish new, revised versions of the GNU Free Documentation License from time to time. Such new versions will be similar in spirit to the present version, but may differ in detail to address new problems or concerns. See \url{http://www.gnu.org/copyleft/}.%
\par
Each version of the License is given a distinguishing version number. If the Document specifies that a particular numbered version of this License ``or any later version'' applies to it, you have the option of following the terms and conditions either of that specified version or of any later version that has been published (not as a draft) by the Free Software Foundation. If the Document does not specify a version number of this License, you may choose any version ever published (not as a draft) by the Free Software Foundation. If the Document specifies that a proxy can decide which future versions of this License can be used, that proxy's public statement of acceptance of a version permanently authorizes you to choose that version for the Document.%
\typeout{************************************************}
\typeout{Paragraphs  11. RELICENSING}
\typeout{************************************************}
\paragraph[{11. RELICENSING}]{11. RELICENSING}\hypertarget{gfdl-section11}{}
``Massive Multiauthor Collaboration Site'' (or ``MMC Site'') means any World Wide Web server that publishes copyrightable works and also provides prominent facilities for anybody to edit those works. A public wiki that anybody can edit is an example of such a server. A ``Massive Multiauthor Collaboration'' (or ``MMC'') contained in the site means any set of copyrightable works thus published on the MMC site.%
\par
``CC-BY-SA'' means the Creative Commons Attribution-Share Alike 3.0 license published by Creative Commons Corporation, a not-for-profit corporation with a principal place of business in San Francisco, California, as well as future copyleft versions of that license published by that same organization.%
\par
``Incorporate'' means to publish or republish a Document, in whole or in part, as part of another Document.%
\par
An MMC is ``eligible for relicensing'' if it is licensed under this License, and if all works that were first published under this License somewhere other than this MMC, and subsequently incorporated in whole or in part into the MMC, (1) had no cover texts or invariant sections, and (2) were thus incorporated prior to November 1, 2008.%
\par
The operator of an MMC Site may republish an MMC contained in the site under CC-BY-SA on the same site at any time before August 1, 2009, provided the MMC is eligible for relicensing.%
\typeout{************************************************}
\typeout{Paragraphs  ADDENDUM: How to use this License for your documents}
\typeout{************************************************}
\paragraph[{ADDENDUM: How to use this License for your documents}]{ADDENDUM: How to use this License for your documents}\hypertarget{gfdl-addendum}{}
To use this License in a document you have written, include a copy of the License in the document and put the following copyright and license notices just after the title page:%
\begin{verbatim}
Copyright (C)  YEAR  YOUR NAME.
Permission is granted to copy, distribute and/or modify this document
under the terms of the GNU Free Documentation License, Version 1.3
or any later version published by the Free Software Foundation;
with no Invariant Sections, no Front-Cover Texts, and no Back-Cover Texts.
A copy of the license is included in the section entitled "GNU
Free Documentation License".
\end{verbatim}
\par
If you have Invariant Sections, Front-Cover Texts and Back-Cover Texts, replace the ``with\dots{} Texts.'' line with this:%
\begin{verbatim}
with the Invariant Sections being LIST THEIR TITLES, with the
Front-Cover Texts being LIST, and with the Back-Cover Texts being LIST.
\end{verbatim}
\par
If you have Invariant Sections without Cover Texts, or some other combination of the three, merge those two alternatives to suit the situation.%
\par
If your document contains nontrivial examples of program code, we recommend releasing these examples in parallel under your choice of free software license, such as the GNU General Public License, to permit their use in free software.%
%
\backmatter
%
\cleardoublepage
\pagestyle{empty}
\vspace*{\stretch{1}}
\centerline{This book was authored in \href{http://mathbook.pugetsound.edu}{PreTeXt}.%
}
\vspace*{\stretch{2}}
\end{document}